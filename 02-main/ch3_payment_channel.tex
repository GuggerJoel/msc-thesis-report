\chapter{Payment channels, a micropayment network}
\label{chap:paymentChannels}

Payment channels or micropayment channels, as mentioned previously, are one part
of the scalability solution. Thus, it exists multiple propositions to construct such
structures. In order to have a better understanding of differences, strengths, and
weaknesses of some of them, a classification is done and properties are defined.

\begin{definition}[trustless]
  A channel is trustless if and only if the funds' safety for every players
  $p_i \in \mathcal{P} = \{\mathcal{P}_0, \dots, \mathcal{P}_n\}$
  at each steps $\mathcal{S}$ of the protocol does not depend on
  players' $\Delta p = \mathcal{P} - p_i$ behavior.
\end{definition}

\begin{definition}[optimal]
  A channel is optimal if and only if the number of transaction $\mathcal{T}(\mathcal{C})$
  needed to claims the funds for a given constraint $\mathcal{C}$ is equal to
  the number of moves $\mathcal{M}(\mathcal{C})$ needed to satisfy the constraint
  at any time without breaking the first definition.

  For a containt $\mathcal{C}$ in a channel $\mathcal{P}_1 \rightarrow \mathcal{P}_2$,
  refunding $\mathcal{P}_1$ where the amount $M \geq 0$ of $\mathcal{P}_2$ require
  $\mathcal{M}(\mathcal{C}) \geq 2$ because of the intermediary revokation's state.
  An optimal scheme require $\mathcal{T}(\mathcal{C}) = \land\mathcal{M}(\mathcal{C}) = 2$.
\end{definition}

\begin{definition}[endless]
  A channel is endless if and only if there is no predetermined lifetime at the setup.
\end{definition}

\begin{definition}[pulseless]
  A channel is pulseless if and only if there is no need to refresh or close the
  channel on-chain while at least one player $p_i \in \mathcal{P} = \{\mathcal{P}_0,
  \dots, \mathcal{P}_n\}$ where the available amount to send is $A(p_i) > 0$. By
  definition a pulseless channel must be also endless.
\end{definition}

\begin{definition}[undelayed]
  A channel is undelayed if and only if each player $p_i \in \mathcal{P} = \{\mathcal{P}_0,
  \dots, \mathcal{P}_n\}$ can trigger their settlement at any time.
\end{definition}

These definitions are used in the following to compare different commonly exposed
payment channel constructions. It is worth noting that the list does not contain all
the payment channel propositions and maybe some of them are missing. However, the
list contain a fairly good representation of the different types existing.

\minitoc

\newpage

% -----------------------------------------------------------------------------
\section{Types of payment channel}

We can distinguish two major type of channels, the unidirectional channel that allow
one user to send money to an other user into a channel and the bidirectional channel
that allow two users to exchange into a channel. Usualy, a bidirectional channel is
more optimize than two unidirectional channels.

\subsection{Unidirectional}

In unidirectional channels there is a payer, hereinafter also player one or client, and
a payee, hereinafter also player two or provider, and it is not possible to transfer
money back in the reverse direction into the channel.

\subsubsection{Spilman-style payment channels}

Spilman-style payment channels, proposed by Jeremy Spilman in 2013 \cite{SpilmanStyle},
are the most simple construction of an unidirectional
payment channel. They have a finite lifetime predefined at the setup phase and the client,
i.e. the payer, cannot trigger its own refund before the end of the lifetime (but he can
receives his funds back if the payee settle the channel before the end of the lifetime).
The channel is one-time use, when the payer or the payee get his funds the channel is
closed. Nor the payer or the receiver need to watch the blockchain to react to event
during the lifetime of the channel because only the payee can broadcast a transaction,
so the payer does not need to watch the blockchain to be safe.

According to the previous definitions Spilman-style payment channels are trustless,
assuming that a suitable solution for transaction malleability has been implemented
\cite{BIP62, DBLP:journals/corr/AndrychowiczDMM13, DBLP:journals/corr/DeckerW14},
optimal but not endless nor undelayed.

\subsubsection{CLTV-style payment channels}

\subsection{Bidirectional}

\subsubsection{Poon-Dryja payment channels}

\subsubsection{Decker-Wattenhofer duplex payment channels}

Decker-Wattenhofer duplex payment channels \cite{Decker2015fast}, also called
\gls{dmc}, propose in 2015 bidirectional channels based on pairs of Spilman-style
unidirectional channels. The construction has a finite lifetime predefined at the setup
phase but can be refresh on-chain to keep the channel open with an updated state. During the
refresh process it is possible to refilled the channel and the scheme allows payment routing
with \gls{htlc}.

Duplex payment channels are trustless and endless, but not optimal (the uncooperative close of the
channel require $d + 2$ transactions), not pulseless (it requires a refresh transaction
when the lifetime run over to keep the channel open), and not undelayed (without cooperation
the funds are recover after $d + 2$ transactions with \texttt{nLockTime} values).

\subsection{Summary}

\begin{table}[h]
  \begin{tabularx}{\textwidth}{ | X | l | l | l | l | l | l |}
  \hline
  Channel & Type & Optimal & Endless & Pulseless & Undelayed \\ \hline
  Spilman-style & Uni & Yes & No & No & No \\ \hline
  CLTV-style & Uni & ? & ? & ? & ? \\ \hline
  Poon-Dryja & Bi & ? & ? & ? & ? \\ \hline
  Decker-Wattenhofer \gls{dmc} & Bi & No & Yes & No & No \\
  \hline
  \end{tabularx}
  \caption{Summary of different payment channels}
  \label{fig:summaryPaymentChannel}
\end{table}

% -----------------------------------------------------------------------------
\section{Our one-way channel}

% -- Your text goes here --

% -----------------------------------------------------------------------------
\section{Optimizing channels}

% -- Your text goes here --
