\chapter{Introduction}
\label{chap:introduction}

Nos programmes sont le plus souvent écrits avec des langages bas niveaux tels le C/C++ qui forcent le développeur à gérer la mémoire lui-même. Ce qui implique que, sans de bonnes connaissances et une attention particulière, un adversaire peut facilement exploiter des bugs qui surviennent au sein de ces mécanismes de gestion. Grâce à cela, l’attaquant peut modifier le control-flow de l'application et exécuter son propre code avec les privilèges donnés au programme ciblé.

Sur les dix dernières années, les attaques de capables de modifier le flôt de contrôle au sein des principaux logiciels que nous utilisons ont augmentées. Etant donné la dangerosité de ce type d’attaque connues depuis cinquante ans (1998 pour le \og grand public \fg) les universités ainsi que les chercheurs en sécurité informatique des grandes entreprises de l'IT (IBM, Intel, Google, Microsoft, etc) ont proposés et mis en place différents concepts de protections visant à empécher ce type spécifique d'attaque. Parmi ces méchanisme de protection on retrouve \gls{aslr}, \gls{dep}/\gls{nx},  \gls{stackCookies}, \og Coarse-grained CFI \fg ou encore \og Finest-grained CFI \fg.

Mais comme à chaque fois, le jeu du chat et de la souris se met en marche et d'autres chercheurs en sécurité parviennent toujours à trouver un moyen de contourner ces mécanismes de protection. Être capable de garantir l'integrité du flôt de contrôle de l'application est un enjeu majeur dans la sécurité des systèmes d'informations d'aujourd'hui.

C'est dans ce contexte qu'un laboratoire de l’\gls{epfl} propose une implémentation appelée Levee qui rassemble des concepts de protection au sein de l’infrastructure de compilation LLVM. L’idée est de séparer les pointeurs jugés sensibles et de les placer dans une zone mémoire sécurisée. La séparation des pointeurs est faite par analyse durant la phase de compilation et permet d’obtenir un cout en performance relativement bas (environ 8\% à 10\%).

Le but de ce rapport est d’expliquer en détail le fonctionnement des concepts de protection sur lequels Levee se base ainsi que d’expérimenter et d’analyser son implémentation. Cependant, pour mieux comprendre les enjeux se cachant derrière ces concepts, un bref récapitulatif du fonctionnement de la mémoire au sein des systèmes d'exploitations modernes ainsi qu'un historique des mécanismes protections et leurs attaques respectives est dressé dans le chapitre suivant.
