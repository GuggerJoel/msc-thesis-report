\chapter{Introduction}
\label{chap:introduction}

Bitcoin is a decentralized peer-to-peer currency that allows users to pay for
things electronically if both parties are willing. There exist thousands of
other crypt-currencies, but only some of them are interesting in a political,
economic or technical point of view. We can also mention Ethereum, ZCash, or
Monero for example. Created by a pseudonymous software developer going by the
name of Satoshi Nakamoto in 2008, as an electronic payment system based on
mathematical proof, Bitcoin as the idea to produce a means of electronic
exchange without any central authority nor censorship, in a secure, verifiable
and immutable way. The blockchain so often mentioned is the output of this
secure, verifiable and immutable mathematical proof.

The most significant challenge in Bitcoin for the coming years is scalability.
Currently, Bitcoin enforces a block-size limit which is equivalent to only some
transactions per second on the network. This amount is not sufficient in
comparison to big payment infrastructures, which allows tens of thousands of
transactions per second and even more in peak times such as Christmas. To
address this, some proposals modifying the transaction structure (such as
SegWit), some proposals also modifying the block-size limit (such as SegWit2x)
and others creating a second layer based on top of the Bitcoin protocol (such as
the Lightning Network) exist. In the same idea of a second layer, this paper
proposes a new implementation of a unidirectional payment channel for retail,
commercial transactions scenario. A unidirectional payment channel allows two
parties to transact over the blockchain while minimizing the number of
transactions needed on the blockchain in a secure and trustless way. Every kind
of channel needs multi-signature addresses to secure the funds. A cryptographic
threshold scheme might improve this schemes significantly. Finding such a
threshold scheme that fulfill the requirements is not trivial. The threshold
scheme selected is adapted and implemented into the Bitcoin cryptographic
library to compute a particular two-party threshold ECDSA signature.

% ECDSA is the signature scheme used by Bitcoin to sign transactions. A standard transaction
% is constituted of a single signature corresponding to the address where the Bitcoins come
% from. But sometimes we need more complex management for locking funds. To address the
% limitation of a single signature, Bitcoin introduced a new OP\_CODE named CHECKMULTISIG with
% a new standard script. With this standard script, it is now possible to spend Bitcoin to an
% address that requires a minimum of m signatures in n authorised signatories and extend the
% capability of Bitcoin to lock funds in a more complex way.
%
% However, some issues appear. The way the script works requires exposing all the public keys
% when an output is signed and this increases the transaction size enormously, which implies
% bigger fees. All the signatures are, obviously, present with the public keys in the transaction
% script, which implies that we can know which public keys signed the transaction. And there
% is some limitation, due to the script size limit, the maximum number of authorised signatories
% is 15. All these limitations mean that we cannot imagine a complex organization nor structure
% with the multi-signature script for the moment.
%
% To address this limitation, a group of researchers published a first paper  in 2015 and a
% second one  in 2016 describing the way to achieve a threshold scheme with DSA and ECDSA.
% Today, there is no well-known implementation ready for production purposes even though
% industries need it. The principal purpose of this thesis is to provide a clear, well
% documented C library, based on the internal ECDSA library present in bitcoin-core.
