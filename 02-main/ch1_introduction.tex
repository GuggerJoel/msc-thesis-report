\chapter{Introduction}
\label{chap:introduction}

Nos programmes sont le plus souvent écrits avec des langages de bas niveau comme le C/C++, qui forcent le développeur à gérer la mémoire lui-même. Cela implique que, sans de bonnes connaissances et une attention particulière, un adversaire peut facilement exploiter des bugs qui surviennent au sein de ces mécanismes de gestion. Grâce à cela, l’attaquant peut modifier le flot de contrôle de l'application et exécuter son propre code avec les privilèges donnés au programme ciblé.

Sur les dix dernières années, le nombre d'attaques capables de modifier le flot de contrôle au sein des principaux logiciels que nous utilisons a augmenté. Etant donné la dangerosité de ce type d’attaques, connues depuis cinquante ans (1998 pour le \og grand public \fg \cite{pharck49, SmashingTheStack}), les universités ainsi que les chercheurs en sécurité informatique des grands groupes (IBM, Intel, Google, Microsoft, etc.) ont proposés et mis en place différents concepts de protection visant à empêcher ce type spécifique d'attaques. Parmi ces mécanismes de protection, on retrouve \gls{aslr}, \gls{dep}/\gls{nx}, les \og \gls{stackCookies} \fg ou encore différentes implémentations de \og \gls{cfi} \fg telles que \og \gls{cg-cfi} \fg et \og \gls{fg-cfi} \fg.

Mais comme à chaque fois, le jeu du chat et de la souris se met en marche et d'autres chercheurs en sécurité parviennent toujours à trouver un moyen de contourner ces mécanismes de protection. Être capable de garantir l'integrité du flot de contrôle de l'application est un enjeu majeur dans la sécurité des systèmes d'informations d'aujourd'hui.

C'est dans ce contexte qu'un laboratoire de l’\gls{epfl} propose une implémentation appelée \gls{levee} qui rassemble des concepts de protection au sein de l’infrastructure de compilation \gls{llvm}. L’idée est de séparer les pointeurs jugés sensibles et de les placer dans une zone mémoire particulière. La séparation des pointeurs est faite par analyse durant la phase de compilation et permet d’obtenir un coût en performance relativement bas (environ 8\% à 10\% de temps d'exécution supplémentaire).

Le but de ce rapport est d’expliquer en détail le fonctionnement des concepts de protection sur lequels \gls{levee} se base et d’expérimenter et analyser son implémentation. Pour mieux comprendre les enjeux se cachant derrière ces concepts, un bref récapitulatif du fonctionnement de la mémoire au sein des systèmes d'exploitations modernes ainsi qu'un historique des mécanismes de protection et des attaques possibles est dressé dans le chapitre suivant.
