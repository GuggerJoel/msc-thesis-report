\chapter{Further research}
\label{chap:furtherResearch}

It is possible to list an enormous number of ideas for further research in a field
like crypto-currencies, blockchain or cryptography. But some of those more related to the work
done in this paper are listed in the following. Some of them are improvements of
the work already done, but not yet ready for production, and some of them are
entirely exploratory.

% -----------------------------------------------------------------------------
\section{Side-channel attack resistant implementation and improvements}

The proposed implementation in the library \texttt{secp256k1} relies upon
\texttt{libgmp} for all mathematical calculus but this library is
not robust against side-channel attacks. The library has not
been developed for that particular purpose. Therefore, another implementation
is needed to handle, in constant time and constant memory if
possible, the mathematical calculus part. This is a significant improvement that
can be done, or must be done, before hoping to use the module in some real case
scenario.

\subsection{Second hash function}

The current implementation uses the hash function \texttt{SHA256} implemented
in the library \texttt{secp256k1} for $\Pi$ and $\Pi'$. This is not compliant
with the scheme requirements. Another hash function must be implemented
and used for $\Pi'$.

\subsection{Paillier cryptosystem}

Two major improvements or modifications can be made specifically on the
Paillier cryptosystem implementation. As shown in Paillier's paper, the
Chinese Remainder Theorem can be used to optimize the decryption. In the
standard approach, with a private key $(n, g, \lambda, \mu)$ and a ciphertext $c
\in \mathbb{Z}_{n^2}^*$ it is possible to compute the plaintext $m =
L(c^{\lambda} \mod n^2) \cdot \mu \mod n$ where $L(x) = \frac{x-1}{n}$. With the
CRT two functions $L_p$ and $L_q$ are defined as

\begin{ceqn}
\begin{align*}
  L_p(x) = \frac{x-1}{p} \quad \text{and} \quad L_q(x) = \frac{x-1}{q}
\end{align*}
\end{ceqn}

Decryption can, therefore, be performed with modulo $p$ and modulo $q$ and
recombining modular residues afterward

\begin{ceqn}
\begin{align*}
  m_p = L_p(c^{p-1} \mod p^2) \ h_p \mod p \\
  m_q = L_q(c^{q-1} \mod p^2) \ h_q \mod q \\
  m = \text{CRT}(m_p, m_q) \mod pq
\end{align*}
\end{ceqn}

with precomputations

\begin{ceqn}
\begin{align*}
  h_p &= L_p(g^{p-1} \mod p^2)^{-1} \mod p \quad \text{and} \\
  h_q &= L_q(g^{q-1} \mod p^2)^{-1} \mod q
\end{align*}
\end{ceqn}

Paillier cryptosystem can also be adapted to \gls{ec} cryptography as shown in the
paper \say{Trapdooring Discrete Logarithms on Elliptic Curves over Rings} by
Pascal Paillier \cite{10.1007/3-540-44448-3_44}. It is worth noting however that
the curve construction is different from the curve used to sign and so the code
base cannot necessarily be reused.

\subsection{Zero-knowledge proofs}

Non-interactive zero-knowledge proofs are a significant research field. The
article \say{From Extractable Collision Resistance to Succinct Non-interactive
Arguments of Knowledge, and Back Again} by Bitansky, Nir and Canetti, Ran and
Chiesa, Alessandro and Tromer, and Eran \cite{Bitansky:2012:ECR:2090236.2090263}
introduced the acronym zk-SNARK for zero-knowledge Succinct Non-interactive
ARgument of Knowledge that is the backbone of the Zcash protocol
\cite{cryptoeprint:2014:349}. In the recent paper \say{Bulletproofs: Efficient
Range Proofs for Confidential Transactions} \cite{cryptoeprint:2017:1066} a new
non-interactive zero-knowledge proof protocol with concise proofs and without a
trusted setup is proposed. Further research could be done to adapt the
zero-knowledge proof construction and migrate to a more generic approach. These
zero-knowledge proofs date from the early 2000s and advancement has been made
since then.

% -----------------------------------------------------------------------------
\section{Hardware wallets}

Hardware wallet devices have become increasingly popular. They promise to keep
the keys safe and, at least, expose the keys less
thanks to a dedicated and controlled environment. Keys can be stored
safely and, in an organization, for example, multiple hardware wallets can be
used to create a multi-signature to control the funds.

The development of this threshold library, even if it is just a 2-out-of-2
multi-signature script equivalent, can be used to create threshold hardware
wallet devices. Two hardware wallet devices can be set up together to create a
multi-user setup, or a hardware wallet device can be coupled with a phone to
secure a web-wallet.

% -----------------------------------------------------------------------------
\section{Key management}

Usually, when a new Bitcoin wallet is created a list of words, called a \textit{mnemonic} phrase,
is shown to the user as a backup of his key. The \textit{mnemonics} are between
twelve and twenty four words, and each word represents 11 bits of the initial seed
\cite{Mnemonic}. For a threshold key, it is not possible to represent all the
data in the same way given the size of the key (near 4.5 Kb). Another way to
display and transmit this information is needed to improve usability. Further
research could be done to find a better way to represent and display a threshold
key.

The master tag is not included in the \texttt{DER} schema. Is the key itself
responsible for storing this information or is this information part of the
setup and can be stored elsewhere? This question can also be explored.

% -----------------------------------------------------------------------------
\section{General threshold scheme}

The way multi-signature scripts work in Bitcoin requires exposing all
public keys related to the signatures. That increases the transaction size,
which implies significant fees. Due to the script size limit of near to 500 bytes,
the maximum number of signatories is around fifteen. The signatures are
naturally present with the public keys in the script, which implies that it is
possible to know which keys signed the transaction. That implies less anonymity
on the blockchain. With a general threshold scheme, these limitations would be
removed.

As previously mentioned, research has been done to generalize and find an
optimal $(t, n)$-threshold in \gls{ecdsa} \cite{10.1007/BFb0052253,
10.1007/978-3-642-27954-6_20}. These papers base their work on the scheme chosen
in this thesis, so a deeper analysis could be performed to assess the
changes needed and adapt the current implementation to construct a generic threshold scheme.

% -----------------------------------------------------------------------------
\section{Schnorr signatures}

In the paper \say{Efficient Identification and Signatures for Smart Cards}
published in CRYPTO 1989, C.P. Schnorr proposes the \say{Schnorr signature
algorithm} \cite{10.1007/0-387-34805-0_22}. The Schnorr signature is considered
the simplest digital signature scheme to be provably secure in a random oracle
model \cite{Bellare:1993:ROP:168588.168596, 10.1007/978-3-642-29011-4_33}.
Bitcoin developers and researchers have had a strong interest in this specific
scheme for some years now. Schnorr signatures could greatly reduce the size of
the signature from 65 bytes (\gls{ecdsa} in \texttt{DER} format) to around 40 bytes.

With the arrival of SegWit, script versioning was also introduced, making it
is easier to introduce a new \texttt{OP\_CODE} and so introduce a new signature
validation scheme. However, this will not invalidate the present work and
research because of the specific requirements needed to optimize payment channels.

Nevertheless, Schorr signatures are tipped to be the next scheme used in Bitcoin
and maybe in other crypto-currencies. Further research could be done to find a
protocol that fulfills the requirements defined for payment channel
optimization.
