\chapter{Conclusions}
\label{chap:conclusions}

Some mechanisms of Bitcoin have been explained to introduce one major
fact in Bitcoin today, i.e., the scalability problem. The problem of scalability
already has existing drafted solutions like consensus changes or layer-two
applications such as payment channels. Payment channels are not new to Bitcoin.
This idea was suggested by Satoshi in an email to Mike Hearn. Since then many
different constructions of unidirectional and bidirectional payment channels
have been discussed. In this thesis a unidirectional payment channel with
specific capabilities is presented, explained, and analyzed.

This layer-two application can be improved with threshold cryptography by
reducing the size of the transactions without changing the security model. This
reduction is feasible by replacing the multi-signature script by a single
signature computed with threshold cryptography. The threshold scheme is analyzed
and adapted to \gls{ecdsa} before being implemented in the existing library used
in Bitcoin-core, the library \texttt{secp256k1}. Finally, further research about
payment channels, Bitcoin in general, and threshold signature schemes are
explored.

The Bitcoin payment channel implementation will be released as open source
software soon, and testing will begin in the coming months. The threshold
implementation will be part of a current project comprising the creation of an
open Bitcoin Teller Machine (BTM) using payment channels to allow instant
cash withdrawal without security risk for the BTM provider and threshold signature
to secure the key on the machine by not keeping the full private key on the machine or backend server.
