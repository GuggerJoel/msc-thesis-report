\chapter{Conclusions}
\label{chap:conclusions}

% -----------------------------------------------------------------------------
\section{Les innovations apportées par Levee}

L'innovation majeur présente au sein de \gls{cpi} est l'approche avec laquelle
les chercheurs ont abordés la problèmatique de protection du flot de contrôle.
Comme montré, il existe déjà des mécanismes tels que SoftBound+CETS,
permettant de se prémunir à 100\% contre les \og control-flow hijack \fg.
Ces mécanismes appliquent les principes des langages \og memory safe \fg sur
l'entiereter du programme, alors que l'approche de \gls{cpi} sélectionne une
partie réduite des pointeurs résponsable du flot de contrôle du programme.
Les résultats obtenu, tant en terme de performance qu'en terme d'efficacité,
sont très bons.

Des améliorations telles que HardBound ou Watchdog réduisant le coût en
performance de SoftBound existent. Ces améliorations repose sur des bases
hardware, permettant d'instrumenter la protection de la manière la plus efficace
possible et de soulager la couche software. Pour \gls{levee}, il en va de même,
il est possible d'augmenter les performances en se basant sur des implémentation
hardware tel que, par exemple, Intel MPX (Memory Protection Extensions)
\cite{IntelMPX}, apparu dans l'architecture Skylake, supportée au niveau
du noyau Linux depuis la version 3.19.

\subsection{\og Safe stack \fg}

\og Safe stack \fg se démarque des \og \gls{stackCookies} \fg également par
son approche, au lieu de vouloir protéger seulement les arguements et les adresses
présentes sur la pile d'exécution, les variables accèdées de manière dangereuse
sont déplacées. Les canaris empêchent déjà le démarage d'une attaque \gls{rop} depuis
la pile, cependant il est possible, sous certaines conditions, de réécrire correctement
le canari et donc de passer outre la protection.

\og Safe stack \fg rend totalement impossible l'exploitation d'un dépassement de
tampon sur la pile par le simple fait que le tampon ne se retrouvera plus sur la pile,
mais dans une zone spéciale présente au sein du segment de \og memory mapping \fg.

% -----------------------------------------------------------------------------
\section{Évaluation des objectifs initiaux}

Les objectifs obligatoires du projet, tel que décrit dans le cahier des charges en
annexe, ont été remplis. Les deux concept \gls{cpi} et \gls{cps} ont été présentés
avec \og \gls{safeStack} \fg. Une description de \gls{llvm} ainsi que l'architecture
de l'implémentation de \og \gls{safeStack} \fg a été rédigée. Un historique des
méthodes de protection ainsi que certaines attaques liées fait office de vue
synoptique des attaques agissant sur le flot de contrôle comme discuté durant
le travail. Deux exemples d'attaque permettant de tester les limites des canaris
ainsi que de \og \gls{safeStack} \fg ont été décrits.

L'objectif optionnel d'analyser les coûts d'un portage de \gls{levee} sur la plateforme
ARM n'a pas été réalisé au moment où ses lignes sont rédigées, bien que la plupart
des éléments nécessaires à cette analyse soit présent dans ce rapport, le temps
a manqué à produire un chapitre ou une section bien documentée concernant les
enjeux à prendre en compte.

% -----------------------------------------------------------------------------
\section{Difficultés rencontrées}

La maîtrise du sujet traité a fait défaut au début du travail, une phase importante
de recherche et de compréhension des éléments sur lesquels se base les
concepts décrits a été nécessaire. C'est aussi pourquoi les sections dédiées aux
récapitulatifs sont aussi fournies pour un rapport de ce type. Aujourd'hui je peux
dire que le sujet en général a bien été parcouru et certains aspets ont été
étudiés plus en profondeur. Les enjeux majeurs et les concepts les plus importants
ont été relayés dans ce rapport avec les explications nécessaires à leur bonne
compréhension.

La mise en place des attaques a été un long chemin parfois difficile. Le chapitre
contient plusieurs attaques documentées afin de représenter la quantité de travail
fournie.

% -----------------------------------------------------------------------------
\section{Sujets de recherche à développer}

On peut distinguer deux parties, \gls{cpi} et \og \gls{safeStack} \fg.
Pour \gls{cpi} les sujets de recherche à développer en premier serait l'intégration
de la technologique Intel MPX, le papier la mentionnait comme étant une technologie
à paraître, maintenant que celle-ci existe il serait interessant de l'étudier de
plus prêt.

Pour \og \gls{safeStack} \fg, le sujet à développer plus en profondeur pourrait
être sa comparaison avec d'autres moyens permettant de se proteger des dépassements
de tampons.
