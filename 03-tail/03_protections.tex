\chapter{Mécanismes de protection}
\label{chap:mecanismeProtection}

Les \og \gls{stackCookies} \fg sont vérifiées lors de l'épilogue de la fonction, on
peut voir sur le \autoref{lst:stackCookiesASM} que la valeur originale ainsi que la
valeur présente sur la pile sont chargées dans les registres \texttt{\%edx} et
\texttt{\%ecx} puis sont comparés. Dans le cas ou le résultat n'est pas égal, la
fonction \texttt{\_\_stack\_chk\_fail@plt} est appelée, dans le cas contraire
l'instruction de retour est exécutée.

\begin{listing}
	\asmfile{02-main/listings/stackCookies.s}
	\caption{Épilogue de fonction vulnérable vérifiant la valeur du canari}
	\label{lst:stackCookiesASM}
\end{listing}

Dans le cas de \og \gls{safeStack} \fg, on peut voir que le pointeur passé en paramètre
de la fonction \texttt{gets()} pointe sur la pile non-sécurisée.

\asmfile{02-main/listings/safeStack.s}
\captionof{listing}{Code de la fonction vulnérable compilée avec \og \gls{safeStack} \fg}
\label{lst:safeStackASM}
