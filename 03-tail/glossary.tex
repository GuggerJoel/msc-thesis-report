% Terms
% -----
% format:  \newglossaryentry{<label>}{<settings>}
% example: \newglossaryentry{computer}
%{
%	name=computer,
%	description={is a programmable machine that receives input,
%		stores and manipulates data, and provides
%		output in a useful format}
%}

% \newglossaryentry{nosql}
% {
% 	name=NoSQL,
% 	description={Database not using the relational model and the \acrshort{sql} language}
% }


% Display ASLR
\newglossaryentry{aslr}
{
	name=ASLR,
	description={Address Space Layout Randomization, technique permettant de rendre aléatoire la position des ségments mémoires}
}

\newglossaryentry{cg-cfi}
{
	name=Coarse-grained CFI,
	description={Coarse-grained CFI est une implémentation simplifiée du principe de Control-Flow integrity, échangeant sécurité contre plus de performances}
}

\newglossaryentry{fg-cfi}
{
	name=Finest-grained CFI,
	description={Finest-grained CFI est une implémentation plus complète du principe de Control-Flow integrity. Garantissant une bonne sécurité mais ayant un coût élevé en performances}
}

\newglossaryentry{dep}
{
	name=DEP,
	description={Data Execution Prevention, technique permettant de marquer un espace vituel de mémoire non-exécutable}
}

\newglossaryentry{nx}
{
	name=NX,
	description={NX bit pour No-eXecute bit est une technique utilisée dans les processeurs pour dissocier les zones de mémoire contenant des instructions des zones contenant des données}
}

\newglossaryentry{stackCookies}
{
	name={Stack cookies},
	description={Les stack cookies, ou stack canaries, sont des valeurs déposées sur la pile d'exécution après la valeur de retour lors de l'appel d'une fonction et son controlées à l'épilogue de la-dite fonction}
}

\newglossaryentry{stackCanaries}
{
	name={Stack canaries},
	description={Les stack canaries sont un synonyme de stack cookies}
}

\newglossaryentry{levee}
{
	name={Levee},
	description={Levee est une implémentation des concepts de protection CPI, CPS et Safe Stack. Actuellement, mai 2017, une partie du projet a été intégré au sein de LLVM sous le nom de Safe Stack}
}

\newglossaryentry{llvm}
{
	name={LLVM},
	description={LLVM, à la base Low Level Virtual Machine et maintenant nom à part entière, est une approche divergente aux compilateur tel que GCC et une collection d'outils de compilation}
}

\newglossaryentry{clang}
{
	name={Clang},
	description={Clang est un compilateur pour les langages de programmation C, C++, Objective-C et Objective-C++. Son interface de bas niveau utilise les bibliothèques LLVM pour la compilation}
}

\newglossaryentry{safeStack}
{
	name={Safe Stack},
	description={Safe Stack est un composant de de CPI/CPS permettant de traiter particulièrement la gestion des pointeurs présents au sein de la pile d'exécution}
}

\newglossaryentry{lldb}
{
	name={LLDB},
	description={LLDB Debugger est présenté comme étant un débegeur de dernière génération et éfficace. Il fait parti des outils développés au sein du plus large projet qu'est LLVM}
}


% Acronyms
% --------
% format:  \newacronym{<label>}{<abbrv>}{<full>}
% example: \newacronym{lvm}{LVM}{Logical Volume Manager}
% plural:  \newacronym[longplural={Frames per Second}]{fpsLabel}{FPS}{Frame per Second}

% % Display Address Space Layout Randomization (ASLR)
% \newacronym{aslr}{ASLR}{Address Space Layout Randomization}


\newacronym{epfl}{EPFL}{École polytechnique fédérale de Lausanne}
\newacronym{nop}{NOP}{No Operation}
\newacronym{cfg}{CFG}{Control-Flow graph}
\newacronym{cfi}{CFI}{Control-Flow integrity}
\newacronym{eof}{EOF}{End Of File}
\newacronym{rop}{ROP}{Return Oriented Programming}
\newacronym{cpi}{CPI}{Code-pointer integrity}
\newacronym{cps}{CPS}{Code-pointer separation}
\newacronym{gdb}{GDB}{The GNU Project Debugger}

% \newacronym{api}{API}{Application Programming Interface}
%
% \newacronym{cep}{CEP}{Complex Event Processing}
% \newacronym{ci}{CI}{Continuous Integration}
% \newacronym{cqrs}{CQRS}{Command Query Responsibility Segregation}
% \newacronym{crud}{CRUD}{Create-Read-Update-Delete}
%
% \newacronym{dag}{DAG}{Directed Acyclic Graph}
% \newacronym{dsl}{DSL}{Domain Specific Language}
%
% \newacronym{eca}{ECA}{Event Condition Action}
% \newacronym{elk}{ELK}{Elasticseach Logstash and Kibana}
% \newacronym{efk}{EFK}{Elasticseach Fluentd and Kibana}
% \newacronym{epa}{EPA}{Event Processing Agent}
% \newacronym{epn}{EPN}{Event Processing Network}
%
% \newacronym{gelf}{GELF}{Graylog Extended Log Format}
% \newacronym{ge}{GE}{Generic Enabler}
%
% \newacronym{ide}{IDE}{Integrated Development Environment}
% \newacronym{iot}{IoT}{Internet of Things}
%
% \newacronym{jar}{JAR}{Java ARchive}
% \newacronym{jmx}{JMX}{Java Management Extensions}
% \newacronym{json}{JSON}{JavaScript Object Notation}
% \newacronym{jvm}{JVM}{Java Virtual Machine}
%
% \newacronym{poc}{PoC}{Proof of Concept}
%
% \newacronym{rest}{REST}{Representational state transfer}
% \newacronym{rest_markup}{reST}{reStructuredText}
% \newacronym{rpc}{RPC}{Remote Procedure Call}
%
% \newacronym{sql}{SQL}{Structured  Query Language}
%
% \newacronym{uuid}{UUID}{Universally Unique Identifier}
% \newacronym{uri}{URI}{Universal Resource Identifier}
